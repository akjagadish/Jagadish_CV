%-------------------------------------------------------------------------------
%	SECTION TITLE
%-------------------------------------------------------------------------------
\cvsection{Research Experience}
%-------------------------------------------------------------------------------
%	CONTENT 
%-------------------------------------------------------------------------------
\begin{cventries}
\vspace{2mm}

    \cventryLab
    {Ph.D. Thesis}
    {University of Tübingen ~·~ Max Planck Institute for Biological Cybernetics  ~·~ Helmholtz Munich} % Institution line
    {Computational Principles of Intelligence Lab ~·~ Advisors: Dr. Marcel Binz and Prof. Eric Schulz}% lab line
    {Tübingen, Germany} % Location 
    {Apr. 2021 - ongoing}
    {% Description
    \begin{cvitems}
    \item \textit{Human Cognition}
     \item ~·~  Proposed a new framework for building computational cognitive models termed as `Meta-learned models of cognition'' \href{https://arxiv.org/abs/2304.06729}{[PDF]}.
     \item ~·~  Demonstrated that resource rational meta-learned inference explains zero-shot compositional reasoning in humans \href{https://osf.io/preprints/psyarxiv/ymve5}{[PDF]}.
      \item ~·~ Illustrated that ecologically rational meta-learned inference explains human category learning \href{https://arxiv.org/abs/2402.01821}{[PDF]}.
      \item ~·~ Showed that humans and large-language models display symmetric belief updating \href{https://arxiv.org/pdf/2402.01821.pdf}{[PDF]}.
      \item ~·~ Developed bounded ecologically rational meta-learned inference to explore the role of bouded resources and ecological priors in human learning \href{}{[In Preparation]}.
      \item ~.~ Contributed towards building the first foundation model of human cognition \href{https://akjagadish.github.io/assets/Centaur_3_1__a_foundation_model_of_human_cognition.pdf}{[PDF]}.
      \item ~·~ Automating congnitive modelling with large language models \href{}{[In Preparation]}.
      \item \hfill 
      \item \textit{Machine Cognition}
      \item ~·~ Illustrated that Inducing anxiety in large language models increases exploration and bias \href{https://arxiv.org/abs/2304.11111}{[PDF]}.
      \item ~·~ Demonstrated that narrative of traumatic experiences increases state anxiety in large language models, but using mindfulness-based techniques can help alleviate the same \href{https://osf.io/j7fwb}{[PDF]}.
      \item ~·~ Used sparse autoencoders to reveal temporal difference learning in large language mdels \href{https://arxiv.org/pdf/2410.01280}{[PDF]}.       
    \end{cvitems}
    }
% %---------------------------------------------------------

    % \cventryLab
    % {Masters Thesis}
    % {Max Planck Institute for Biological Cybernetics } % Institution line
    % {Computational Principles of Intelligence Lab ~·~ Advisors: Dr. Marcel Binz and Prof. Eric Schulz}% lab line
    % {Tübingen, Germany} % Location 
    % {Jun. 2020 - Dec. 2020}
    % {% Description
    % \begin{cvitems}
    %  \item ~·~ Built meta-reinforcement learning agents that can discover and compose latent structures underlying rewards of structured bandit tasks, and further, generalize to novel structures unseen during training.
    % \end{cvitems}
    % }
% %---------------------------------------------------------

  \cventryLab
    {Graduate Research Assistant} % Position
    {University of Tübingen} % Institution line 1
    {Sinz Lab ~·~ Advisors: Prof. Fabian Sinz and Prof. Edgar Walker} % Institution line 2
    {Tübingen, Germany} % Location 
    {Nov. 2018 - Mar. 2021} % Date(s)
    { % Description
      \begin{cvitems}
      \item ~·~ Built a factor analysis model on top of a convolutional neural network, which predicts stimulus-based neural activity, to recover non-stimulus-related latent brain states \href{https://proceedings.neurips.cc/paper/2021/file/84a529a92de322be42dd3365afd54f91-Paper.pdf}{[PDF]}. 
      \item ~·~ Developed a novel parameter-efficient readout, called a Gaussian readout, that maps nonlinear features learned by the deep convolutional network to the response of each neuron \href{ https://openreview.net/pdf?id=Tp7kI90Htd}{[PDF]}.
      \end{cvitems}
    }

%---------------------------------------------------------
  \cventryLab
    {Graduate Research Assistant} % Position
    {Max Planck Institute for Biological Cybernetics} % Institution line
    {Computational Neuroscience Lab ~·~ Advisor: Prof. Peter Dayan}% lab line
    {Tübingen, Germany} % Location 
    {Nov. 2019 - Feb. 2020}
    {% Description
    \begin{cvitems}
    \item ~·~ Conducted a literature review on the role of Dopamine in reward-based learning \href{https://drive.google.com/file/d/1OEo0gj_9koHtPa58SKOdccETzw48jGHZ/view?usp=sharing}{[PDF]}. 
    \item ~·~ Analyzed behavior (choices and reaction times) of monkeys, whose dopamine receptors were pharmacologically stimulated, during a rule-based categorization task \href{https://drive.google.com/file/d/1bqvoALW3b3Ovm6vpJbzl-hcy932nN__t/view?usp=sharing}{[PDF]}. 
    \end{cvitems}
    }

%---------------------------------------------------------
  \cventryLab
    {AI Researcher} % Position
    {Wadhwani Institute for Artificial Intelligence} % Institution line 1
    {AI for Social Impact ~·~ Advisor: Dr. Rahul Panicker} % Institution line 2
    {Mumbai, India} % Location
    {May 2018 - Sep. 2018} % Date(s)
    { % Description
      \begin{cvitems}
      \item ~·~ Developed a model based on deep-learning that predicts weight of an object from its images [NDA].
      \end{cvitems}
    }
    
%---------------------------------------------------------
  \cventryLab
    {Postbaccalaureate Research Assistant} % Position
    {University of Minnesota, Twin-cities} % Institution line 1
    {Computational Visual Neuroscience Lab ~·~ Advisor: Prof. Kendrick Kay} % Institution line 2
    {Minnesota, USA} % Location
    {Jul. 2017 - Feb. 2018} % Date(s)
    { % Description
      \begin{cvitems}
      \item ~·~  Developed a generative model that factors in the bottom-up, stimulus-driven, and top-down, goal-driven attentional state to characterize cortical fMRI responses for various stimuli and attentional loci combinations. 
      \end{cvitems}
    }
    
%---------------------------------------------------------
  \cventryLab
    {Undergraduate Thesis} % Position
    {Ecole Polytechnique Federale de Lausanne} % Institution line 1
    {Medical Image Processing Lab ~·~ Advisors: Prof. D. van de Ville and Prof. P. Giannakopoulos} % Institution line 2  Dr Djalel Meskaldji, Prof. Sven Haller, 
    {Lausanne, Switzerland} % Location
    {Aug. 2016 - May 2017} % Date(s)
    { % Description
      \begin{cvitems}
		\item ~·~ Investigated the relationship between structural and functional connectivity measures derived from MRI, and Neuroticism Extroversion Openness Personality Inventory-Revised (NEOPI) personality traits \href{https://doi.org/10.3389/fpsyg.2018.02652}{[PDF]}.
      \end{cvitems}
    }

%---------------------------------------------------------
  \cventryLab
    {Undergraduate Research Assistant} % Position
    {Indian Institute of Science} % Institution line 1
    {Computational Tomography Lab ~·~ Advisor: Prof. Kasi Rajgopal} % Institution line 2
    {Bangalore, India} % Location
    {May 2015 - May 2017} % Date(s)
    { % Description
      \begin{cvitems}
      \item ~·~ Developed an algorithm, called k-ABC, based on the artificial bee colony algorithm to come up with an optimal variable density sampling scheme for the compressed sensing-based reconstruction of Magnetic Resonance (MR) images \href{https://doi.org/10.1109/TENCON.2016.7848212}{[PDF]}.
      % \item ~·~ Developed a projection method for \textit{k}-space trajectory generation that provides a significant improvement in the image reconstruction quality with a minimal trade-off in terms of scan-time \href{https://drive.google.com/file/d/1BKM6UuAf0kWOX9ty-QN1Yodb6Acpofz7/view?usp=sharing}{[report]}.
      \end{cvitems}
    }


%---------------------------------------------------------
\end{cventries}

