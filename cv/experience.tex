%-------------------------------------------------------------------------------
%	SECTION TITLE
%-------------------------------------------------------------------------------
\cvsection{Research Experience}
%-------------------------------------------------------------------------------
%	CONTENT 
%-------------------------------------------------------------------------------
\begin{cventries}
\vspace{2mm}

    \cventryLab
    {Ph.D. Thesis}
    {Max Planck Institute for Biological Cybernetics } % Institution line
    {Computational Principles of Intelligence Lab ~·~ Advisors: Dr. Marcel Binz and Prof. Eric Schulz}% lab line
    {Tübingen, Germany} % Location 
    {Apr. 2021 - ongoing}
    {% Description
    \begin{cvitems}
     \item ~·~  Proposed the framework of meta-learned models of cognition \href{https://arxiv.org/abs/2304.06729}{[PDF]}
     \item ~·~  Demonstrated that resource rational meta-learned inference explains zero-shot compositional reasoning in humans. \href{https://osf.io/preprints/psyarxiv/ymve5}{[PDF]}
      \item ~·~ Illustrated that ecologically rational meta-learned inference explains human category learning. \href{https://arxiv.org/abs/2402.01821}{[PDF]}
      \item ~·~ Used the framework of meta-reinforcement learning for rational analysis of optimism bias in both humans and Large-language models. \href{https://arxiv.org/abs/2402.03969}{[PDF]}
    \end{cvitems}
    }
% %---------------------------------------------------------

    % \cventryLab
    % {Masters Thesis}
    % {Max Planck Institute for Biological Cybernetics } % Institution line
    % {Computational Principles of Intelligence Lab ~·~ Advisors: Dr. Marcel Binz and Prof. Eric Schulz}% lab line
    % {Tübingen, Germany} % Location 
    % {Jun. 2020 - Dec. 2020}
    % {% Description
    % \begin{cvitems}
    %  \item ~·~ Built meta-reinforcement learning agents that can discover and compose latent structures underlying rewards of structured bandit tasks, and further, generalize to novel structures unseen during training.
    % \end{cvitems}
    % }
% %---------------------------------------------------------

  \cventryLab
    {Research Assistant} % Position
    {Eberhard Karls University of Tübingen} % Institution line 1
    {Sinz Lab ~·~ Advisors: Prof. Fabian Sinz and Prof. Edgar Walker} % Institution line 2
    {Tübingen, Germany} % Location 
    {Nov. 2018 - Mar. 2021} % Date(s)
    { % Description
      \begin{cvitems}
      \item ~·~ Built a factor analysis model on top of a convolutional neural network, which predicts stimulus-based neural activity, to recover non-stimulus-related latent brain states \href{https://proceedings.neurips.cc/paper/2021/file/84a529a92de322be42dd3365afd54f91-Paper.pdf}{[PDF]}. 
      \end{cvitems}
    }

%---------------------------------------------------------
  \cventryLab
    {Research Assistant} % Position
    {Max Planck Institute for Biological Cybernetics} % Institution line
    {Computational Neuroscience Lab ~·~ Advisor: Prof. Peter Dayan}% lab line
    {Tübingen, Germany} % Location 
    {Nov. 2019 - Feb. 2020}
    {% Description
    \begin{cvitems}
    \item ~·~ Conducted a literature review on the role of Dopamine in reward-based learning for the essay rotation \href{https://drive.google.com/file/d/1OEo0gj_9koHtPa58SKOdccETzw48jGHZ/view?usp=sharing}{[PDF]}. 
    \item ~·~ Analyzed behaviour (responses and reaction times) of monkeys, whose DA receptors were being stimulated, during a rule-based decision task for the laboratory rotation \href{https://drive.google.com/file/d/1bqvoALW3b3Ovm6vpJbzl-hcy932nN__t/view?usp=sharing}{[PDF]}. 
    %We looked at the change in proportion and type of responses, and reaction times (RTs) for each over course of the session specifically, during drug infusion periods, and performed drift-diffusion modelling (DDM) on RTs to characterize the influence of fatigue and drugs on decision threshold and evidence accumulation.s
    \end{cvitems}
    }

%---------------------------------------------------------
  \cventryLab
    {AI Researcher} % Position
    {Wadhwani Institute of Artificial Intelligence} % Institution line 1
    {Advisor: Dr. Rahul Panicker} % Institution line 2
    {Mumbai, India} % Location
    {May 2018 - Sep. 2018} % Date(s)
    { % Description
      \begin{cvitems}
      \item ~·~ Performed measurements, in real-world units, on a 3D model (point cloud/mesh) generated from smartphone images using deep neural networks.
      \end{cvitems}
    }
    
%---------------------------------------------------------
  \cventryLab
    {Research Assistant} % Position
    {University of Minnesota, Twin-cities} % Institution line 1
    {Computational Visual Neuroscience Lab ~·~ Advisor: Prof. Kendrick Kay} % Institution line 2
    {Minnesota, USA} % Location
    {Jul. 2017 - Feb. 2018} % Date(s)
    { % Description
      \begin{cvitems}
      \item ~·~  Worked on developing a generative model that factors in the bottom-up, stimulus-driven, and top-down, goal-driven attentional state to characterize cortical fMRI responses for various stimuli and attentional loci combinations. 
      \end{cvitems}
    }
    
%---------------------------------------------------------
  \cventryLab
    {Bachelor Thesis} % Position
    {Ecole Polytechnique Federale de Lausanne} % Institution line 1
    {Medical Image Processing Lab ~·~ Advisors: Prof. Dimitri van de Ville and Prof. P. Giannakopoulos} % Institution line 2  Dr Djalel Meskaldji, Prof. Sven Haller, 
    {Lausanne, Switzerland} % Location
    {Aug. 2016 - May 2017} % Date(s)
    { % Description
      \begin{cvitems}
		\item ~·~ Studied relationship between structural and functional connectivity measures derived from MRI, and Neuroticism Extroversion Openness Personality Inventory-Revised personality traits \href{https://doi.org/10.3389/fpsyg.2018.02652}{[PDF]}.
      \end{cvitems}
    }

%---------------------------------------------------------
  \cventryLab
    {Research Assistant} % Position
    {Indian Institute of Science} % Institution line 1
    {Computational Tomography Lab ~·~ Advisor: Prof. Kasi Rajgopal} % Institution line 2
    {Bangalore, India} % Location
    {May 2015 - May 2017} % Date(s)
    { % Description
      \begin{cvitems}
      \item ~·~ Developed an algorithm, called k-ABC, based on the artificial bee colony algorithm to come up with an optimal variable density sampling scheme for the compressed sensing-based reconstruction of Magnetic Resonance (MR) images \href{https://doi.org/10.1109/TENCON.2016.7848212}{[PDF]}.
      % \item ~·~ Developed a projection method for \textit{k}-space trajectory generation that takes into account the location and energy content of samples in the 2D frequency space and provides a significant improvement in the image reconstruction quality with a minimal trade-off in terms of scan-time \href{https://drive.google.com/file/d/1BKM6UuAf0kWOX9ty-QN1Yodb6Acpofz7/view?usp=sharing}{[report]}.
      \end{cvitems}
    }


%---------------------------------------------------------
\end{cventries}

