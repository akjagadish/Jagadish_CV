%!TEX TS-program = xelatex
%!TEX encoding = UTF-8 Unicode
% Awesome CV LaTeX Template for Cover Letter
%
% This template has been downloaded from:
% https://github.com/posquit0/Awesome-CV
%
% Authors:
% Claud D. Park <posquit0.bj@gmail.com>
% Lars Richter <mail@ayeks.de>
%
% Template license:
% CC BY-SA 4.0 (https://creativecommons.org/licenses/by-sa/4.0/)
%
%-------------------------------------------------------------------------------
% CONFIGURATIONS
%-------------------------------------------------------------------------------
% A4 paper size by default, use 'letterpaper' for US letter
\documentclass[11pt, a4paper, academicons]{awesome-cv}
\usepackage{academicons}

% Configure page margins with geometry
\geometry{left=1.4cm, top=1.4cm, right=1.4cm, bottom=1.8cm, footskip=.5cm}

% Specify the location of the included fonts
\fontdir[fonts/]

% Color for highlights
% Awesome Colors: awesome-emerald, awesome-skyblue, awesome-red, awesome-pink, awesome-orange
%                 awesome-nephritis, awesome-concrete, awesome-darknight
\colorlet{awesome}{awesome-red}
% Uncomment if you would like to specify your own color
% \definecolor{awesome}{HTML}{CA63A8}

% Colors for text
% Uncomment if you would like to specify your own color
% \definecolor{darktext}{HTML}{414141}
% \definecolor{text}{HTML}{333333}
% \definecolor{graytext}{HTML}{5D5D5D}
% \definecolor{lighttext}{HTML}{999999}

% Set false if you don't want to highlight section with awesome color
\setbool{acvSectionColorHighlight}{true}

% If you would like to change the social information separator from a pipe (|) to something else
\renewcommand{\acvHeaderSocialSep}{\quad\textbar\quad}


%-------------------------------------------------------------------------------
%	PERSONAL INFORMATION
%	Comment any of the lines below if they are not required
%-------------------------------------------------------------------------------
% Available options: circle|rectangle,edge/noedge,left/right

\name{Wy Ming}{Lin}
\position{Masters Student in Neural and Behavioural Sciences}

\mobile{+49 176 87849438}
\email{wlin7439@gmail.com}
\linkedin{WyMingLin}
\rg{Wy\_Ming\_Lin}


%-------------------------------------------------------------------------------
%	LETTER INFORMATION
%	All of the below lines must be filled out
%-------------------------------------------------------------------------------
% The company being applied to
\recipient
  {Application for a Doctoral Position at the Max Planck School of Cognition}
  {}
% The date on the letter, default is the date of compilation
\letterdate{\today}
% The title of the letter
% \lettertitle{}
% How the letter is opened
\letteropening{Dear Selection Committee:}
% How the letter is closed
\letterclosing{Sincerely,}
% Any enclosures with the letter
% \letterenclosure[Attached]{Curriculum vitae, Masters thesis plan, Copies of academic degrees and transcripts, List of publications}


%-------------------------------------------------------------------------------
\begin{document}

% Print the header with above personal informations
% Give optional argument to change alignment(C: center, L: left, R: right)
\makecvheader[R]

% Print the footer with 3 arguments(<left>, <center>, <right>)
% Leave any of these blank if they are not needed
\makecvfooter
  {\today}
  {Wy Ming Lin~~~·~~~Cover Letter}
  {}

% Print the title with above letter informations
\makelettertitle

%-------------------------------------------------------------------------------
%	LETTER CONTENT
%-------------------------------------------------------------------------------
\begin{cvletter}
What drives me to do neuroscience research is knowing that the work I am doing can potentially help others in need. I found my niche while doing a lab rotation at the University Hospital’s Department of Psychiatry and Psychotherapy in Tübingen, where I learned about computational psychiatry as a young field that aims to provide better treatment outcomes to patients. I would love to make contributions in the field by working with the excellent researchers who are part of the Max Planck Society network, whose research I am confident I will add to with my strong skill set, sound theoretical base, and passion for making a positive impact.

I envision my Ph.D. work to address questions of how computational methods can be applied in a neuropsychiatric setting, especially to model symptoms related to affective disorders such as depression. I began getting experience in the field during my first lab rotation with Dr. Nils Kroemer as part of my Masters program at the Graduate Training Centre of Neuroscience in Tübingen. His lab aims to use statistical and computational methods to investigate motivation and its neurobiological mechanisms and to translate these findings to help psychiatric patients. My project involved analyzing the temporal dynamics of effort allocation in a task in which participants exerted physical effort for monetary and food rewards. My goal in the project was to determine how inter-individual variability across time can help characterize healthy participants based on typical effortful behavior. I will continue my work in Dr. Kroemer’s lab in the fourth semester for my thesis, in which I will apply these methods to data from this effort allocation task that will be collected from a group of patients with depression. Through this experience, I learned to use computational tools and built a theoretical basis in psychiatric research, which gave me a good grasp of the current discourse in the field.

The Max Planck School of Cognition would be my ideal next step in working in computational psychiatry. The curriculum and program structure will provide me with the opportunity to learn from a diverse set of researchers during the lab rotation phase and allow me to build on the necessary skills that are required in such a young and interdisciplinary field. First I would like to work with Dr. Elisabeth Binder to learn about the neurological basis of mood and anxiety disorders. Her project on characterizing patients across psychopathologies and identifying common neurobiological signatures is relevant to my interests since modeling these similarities is of particular importance in computational psychiatry. Next, I would like to work with Dr. Ralph Hertwig to develop my theoretical knowledge of decision making processes, while a project with Dr. Hauke Heekeren would show me how these can be investigated with neuroimaging and how they are reflected in psychiatry. Finally, working with Dr. Simon Eickhoff would show me how to bring what I learned together as I use computational methods to model depression. These rotations will allow me to build my background knowledge and give me the practical experiences to successfully complete the doctoral phase of the program, in which my plan would be to provide computational models of affective symptoms to predict effective treatments. 

While my background is in the perception of music, language, and speech, my experiences in these fields helped me develop a strong skill set that can be applicable to work in any area of neuroscience. In my Bachelor’s work with Dr. Psyche Loui, I comprehensively designed and executed an experiment and presented my findings at conferences, including the Northeast Music Cognition Group’s meeting at Harvard University. Working with Dr. Sammler at the MPI added strong programming skills in MATLAB and stimulus design to my repertoire. In my full-time position as a Junior Laboratory Associate in Dr. David Poeppel’s lab, I learned about the publication process through co-authoring a paper in \textit{Nature Neuroscience}. These experiences deepened my understanding of how neuroscience research is done and how to think critically about the questions we pose to investigate the brain.

In summary, I believe that I am a strong candidate for the Max Planck School of Cognition, where my extensive experience in research and a passion for helping others will translate to a successful doctoral experience. I have attached my CV for further review and I look forward to hearing back from the Selection Committee. Thank you for your time and careful consideration. 



\end{cvletter}


%-------------------------------------------------------------------------------
% Print the signature and enclosures with above letter informations
\makeletterclosing
\end{document}
